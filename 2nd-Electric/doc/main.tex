\documentclass[11pt,letterpaper,english]{article}

\everymath=\expandafter{\the\everymath\displaystyle}
\usepackage[margin=1.0in]{geometry}
\usepackage{graphicx}
\usepackage{upgreek}

\title{Regression Analysis of Electric Data}
\author{Ian Clark}
\date{}


\begin{document}

\maketitle

\section*{Problem 1}
\begin{center}
$\hat{y} = -70.4951 + 4.7311\upbeta_O + 0.0036\upbeta_L + 0.28\upbeta_K + 0.7835\upbeta_F$
\end{center}

\section*{Problem 2}
Ceteris paribus, each number ($4.73$, $.0036$, \&c.) show the amount that $\hat{y}$ will increase per unit/percentage increase in the $\upbeta$. Labor has the smallest effect on overall price while desired output will increase cost the most. Signs of the variables are covered in the answer to problem 3. With a ^$R^2$ of $0.9438$ and an adjusted $R^2$ this model seems to be a good fit.

\vspace{5mm}
\noindent
$\upbeta_O$ (or the estimate of Output) is the predominate driver within this regression model. At a value of $4.7311$, output leads the cost estimate--not including the overhead costs.

\section*{Problem 3}
The sign of the output ($\upbeta_O$) makes perfect sense, as overhead costs exist, so whenever the firm is not producing electricity, they will be suffering losses.

\vspace{5mm}
\noindent
For the price variables, as well, the signs make perfect sense because a negative sign on labor, capital, or fuel (when producing energy) would make no sense. You don't have workers paying to work for you and negative consumption of fuel is impossible).

\section*{Problem 4}
The parameter estimate of \textbf{Output} shows that desired or planned output is one of the strongest (in terms of magnitude) in changing the operating price. That is, the higher the Output, the higher the price.

\begin{center}
$\upepsilon_{{\hat{Y}},F} = \frac {\mathit{\Updelta {\hat{Y}}}} {\mathit{\Updelta F}} * \frac F {\hat{Y}} = \upbeta_F * \frac F {\hat{Y}} = 0.7835 * \frac {30.56} {44.219} = 1.335$
\end{center}

The Elasticity of Cost w.r.t is equal to 1.335. This means that for every $1\%$ increase/decrease in fuel will lead to a $1\%$ increase/decrease in Cost.

\clearpage

\section*{Problem 5}
Please note: Manual work included along-side software output\\
$t_{119, 0.025} = 1.98$\\
$0.2801 \pm 1.98 * 0.1295$\\
$0.2801 \pm 0.2564$\\
$[0.0231 \rightarrow 0.5365]$

\section*{Problem 6}
$H_0: Labor \leq 0$\\
$H_1: Labor \geq 0$

\vspace{5mm}
\noindent
$\mathit{CV} = 2.358$\\
$\mu = {\frac {0.0036} {0.0011}} = 3.4379$\\
$3.3479 > 2.358$\\

\noindent
Therefore, we must reject the null hypothesis. The The labor paramater is non-negative

\section*{Problem 7}
$H_0: R^2 = 0$\\
$H_1: R^2 \neq 0$

\vspace{5mm}
\noindent
$CV = 2.4472 (\alpha = 0.05, DF = 4 / 118)$\\
\vspace{5mm}
$F = \frac {\frac {0.9438} 4} {(1-{0.9438}) * 118}$\\
\vspace{5mm}
$0.0356 < 2.4472$

\noindent
Therefore, we fail to reject the null hypothesis.

\end{document}
