\documentclass[11pt,letterpaper,english]{article}

\usepackage[margin=1.0in]{geometry}
\usepackage{helvetica}
\usepackage{graphicx}
\usepackage{stmaryrd}

\newcommand{\newpar}{\vspace{10mm}\noindent}

\title{Paper 4: An Analysis of Coca Cola}
\author{Ian Clark}
\date{}

\begin{document}
\maketitle

\section*{The Discount Rate}
To choose the discount rate, I have opted to use the ten-year treasury as the guide. I am choosing
this rate for the following: I feel that the ten-year treasury (and to a large extent the Federal
Reserve's actions in relation to it) allow a rational layman to reasonably estimate general economic
conditions. So, my current Discount Rate will be $1.92$ as of today. However, since the ten-year treasury
rate has been so volatile, I have decided to use the range of $\{1.90 \shortrightarrow 1.99\}$ for 
further calculations, so to give a more reasonable estimate.

\section*{Market Risk Premium}
In determining the Market Risk Premium, I have chosen to follow the implied approach. Because I have chosen
this method of determining the MRP, I must establish the following: current dividend yield of $1.91$.

\newpar
Now, the expected earnings growth and the Expected Return on Stocks must be estimated using these formulas:

\begin{center}
$Expected Earnings Growth = Productivity + Labor Growth + Inflation$
\end{center}

\begin{center}
$ExpectedReturnOnStocks = CurrDivYield + ExpectedEarningsGrowth$
\end{center}

\begin{center}
$MarketRiskPremium = ExpectedReturnOnStocks - RiskFreeRate$
\end{center}

\noindent
Firstly, I will establish Inflation as the $\%\Delta{Core CPI}$. I have chosen the Core CPI because
it better covers the current economic state. The core CPI is the CPI less food and energy and because
of the volatility within the energy market, specifically due to the competition introduced within the oil
market. Moreover, I don't feel the in-elasticity introduced by food in the CPI is necessary. So, we can 
assert that $\Pi = 1.8$. 

\newpar
Next is Labor Growth and Productivity. I don't have any other fancy metrics
for these and Professor Sweet's reasoning behind his numbers seemed solid, so I will follow those.
So, $LaborGrowth = .5\%$ and $Productivity = 2\%$.

\newpar
So, with our newly established variables, it can be concluded that:

\begin{center}
$Expected Earnings Growth = .02 + .005 + .18 = .43$
\end{center}

\begin{center}
$ExpectedReturnOnStocks = 0.019 + 0.043 = .062$
\end{center}

\begin{center}
$MarketRiskPremium = .062 - .0191 = .0429$
\end{center}

\section*{Determination of Beta}

\subsection*{Revenue Sensitivity}
As surprising as this sounds, Coca-Cola is the second-most recognized word in the world, following OK.
Moreover, there can be argued a great in-elasticity between Coke and other soft drink products 
because consumers of Coke are unlikely to switch to Pepsi and vice-versa. As such, it is fair to
assert that within recession, Coca-Cola can continue to prosper. Because of this, it is a fair 
assertion that Coca-Cola's stock will be below 1.

\subsection*{Operating Leverage}
Coca-Cola has high fixed costs mostly due to research \& development and production. As we can see in
the following two tables, Coca-Cola ranks lower than the median by a significant margin in both Gross 
and Net margins which can imply that the stock will be below 1 in terms of Beta.

\begin{table}[h]
\centering
\begin{tabular}{|c|c|c|c|c|}
\hline
Gross Margin Standard Deviation  &  & S\&P 500 & KO UN Equity & Rank \\ \hline
Median     &  & 2.88     & 1.73         & 20.6 \\ \hline
90 Percentile   &  & 8.93     &              &      \\ \hline
10th Percentile &  & 1.00     &              &      \\ \hline
\end{tabular}
\end{table}

\begin{table}[h]
\centering
\begin{tabular}{|c|c|c|c|c|}
\hline
Net Margin Standard Deviation  &  & S\&P 500 & KO UN Equity & Rank \\ \hline
Median     &  & 3.26     & 1.46         & 14.6 \\ \hline
90 Percentile   &  & 11.42     &              &      \\ \hline
10th Percentile &  & 1.11     &              &      \\ \hline
\end{tabular}
\end{table}

\subsection*{Financial Leverage}
As the table below depicts, Coca-Cola has a significantly high debt in relation to the S\&P 500.
There has been a trend of estimating the beta to be below one and I do not think this high debt
will do much to change it. Even though, as depicted below, Coca-Cola has massive amounts of debt.

\begin{table}[h]
\centering
\begin{tabular}{|c|c|c|c|c|}
\hline
Debt to Equity  &  & S\&P 500 & KO UN Equity & Rank \\ \hline
Median          &  & 61.90    & 111.71       & 75.6 \\ \hline
90 Percentile   &  & 200.70   &              &      \\ \hline
10th Percentile &  & 13.60    &              &      \\ \hline
\end{tabular}
\end{table}

\subsection*{CAP-M K_e}
Before any calculations are completed, I would assume that Coca-Cola's beta is in the range of
$\{0.5 \shortrightarrow 0.7\}$. However, we're not not (entirely) in the business of estimations,
so I will assert that beta will be 0.7. Well below 1, but not too low.

\begin{center}
$K_e = 1.91 + .7 * 1.91 = 4.98\%$
\end{center}


\newpage
\Large\Large\Large\begin{center}
Paper 5: A Valuation of Coca Cola
\end{center}

\section*{Gordon Growth Model}
The biggest assumption here will be that dividends growth is relative to the growth in earnings. I
am making this decision to calculate it this way because I am more interested in the long term 
growth of a company. So, our long term growth will be the same as our expected earnings from paper 
four, or $4.3\%$.

\begin{center}
$Value = 1.32 * (1 + 0.043) / (0.0498 - 0.043) = \$ 202.2467$
\end{center}


\section*{Capitalized Earnings Model}
Given my justification for Core CPI in paper 4, I will choose to use a $2\%$ for inflation, as it 
is nearer to my original estimate. From our calculations, we can assert that Coca-Cola is undervalued
by nearly $25\%$ from its 24 April 2015 closing price of $41.04$.

\begin{center}
$Value = 1.58 / (4.98\% - 2\%) = 53.02$
\end{center}


\section*{H Model}
Given our previous estimations, we have two remaining estimates: H and Short-term growth. For the short-
term growth, I have opted to use the PEG ratio.

\begin{center}
$ShortTermGrowth = (41.04 / 1.58) / 4.22 = 6.035\%$
\end{center}

\begin{center}
$Value = (1.32*(1+0.043)+1.32*3*(0.06035-0.043))/(0.0498-0.043) = \$ 212.56$
\end{center}


\section*{Conclusions}
Needless to say, I don't think that the Gordon Growth Model offers a viable stock price. The Capitalized
Earnings Model, however, offers a reasonable price, and a price that is undervalued and warrants
a purchase. With the H-model, as well, one would certainly purchase at that price.


\newpage
\begin{thebibliography}{9}
\bibitem{Yahoo Finance}
    Yahoo! Finance,
    \emph{$http://finance.yahoo.com/q?s=KO$}.
\bibitem{Yahoo Finance}
    Yahoo! Finance,
    \emph{$http://finance.yahoo.com/q?s=SPY$}.
\bibitem{BLS}
    Bureau of Labor Statistics,
    \emph{$http://www.bls.gov/news.release/cpi.nr0.htm$}.
\end{thebibliography}

\end{document} 